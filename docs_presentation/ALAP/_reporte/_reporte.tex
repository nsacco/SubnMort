\documentclass[12pt,spanish,]{article}
\usepackage{lmodern}
\usepackage{amssymb,amsmath}
\usepackage{ifxetex,ifluatex}
\usepackage{fixltx2e} % provides \textsubscript
\ifnum 0\ifxetex 1\fi\ifluatex 1\fi=0 % if pdftex
  \usepackage[T1]{fontenc}
  \usepackage[utf8]{inputenc}
\else % if luatex or xelatex
  \ifxetex
    \usepackage{mathspec}
  \else
    \usepackage{fontspec}
  \fi
  \defaultfontfeatures{Ligatures=TeX,Scale=MatchLowercase}
\fi
% use upquote if available, for straight quotes in verbatim environments
\IfFileExists{upquote.sty}{\usepackage{upquote}}{}
% use microtype if available
\IfFileExists{microtype.sty}{%
\usepackage{microtype}
\UseMicrotypeSet[protrusion]{basicmath} % disable protrusion for tt fonts
}{}
\usepackage[margin=1in]{geometry}
\usepackage{hyperref}
\hypersetup{unicode=true,
            pdfborder={0 0 0},
            breaklinks=true}
\urlstyle{same}  % don't use monospace font for urls
\ifnum 0\ifxetex 1\fi\ifluatex 1\fi=0 % if pdftex
  \usepackage[shorthands=off,main=spanish]{babel}
\else
  \usepackage{polyglossia}
  \setmainlanguage[]{spanish}
\fi
\usepackage{longtable,booktabs}
\usepackage{graphicx,grffile}
\makeatletter
\def\maxwidth{\ifdim\Gin@nat@width>\linewidth\linewidth\else\Gin@nat@width\fi}
\def\maxheight{\ifdim\Gin@nat@height>\textheight\textheight\else\Gin@nat@height\fi}
\makeatother
% Scale images if necessary, so that they will not overflow the page
% margins by default, and it is still possible to overwrite the defaults
% using explicit options in \includegraphics[width, height, ...]{}
\setkeys{Gin}{width=\maxwidth,height=\maxheight,keepaspectratio}
\IfFileExists{parskip.sty}{%
\usepackage{parskip}
}{% else
\setlength{\parindent}{0pt}
\setlength{\parskip}{6pt plus 2pt minus 1pt}
}
\setlength{\emergencystretch}{3em}  % prevent overfull lines
\providecommand{\tightlist}{%
  \setlength{\itemsep}{0pt}\setlength{\parskip}{0pt}}
\setcounter{secnumdepth}{0}

%%% Use protect on footnotes to avoid problems with footnotes in titles
\let\rmarkdownfootnote\footnote%
\def\footnote{\protect\rmarkdownfootnote}

%%% Change title format to be more compact
\usepackage{titling}

% Create subtitle command for use in maketitle
\newcommand{\subtitle}[1]{
  \posttitle{
    \begin{center}\large#1\end{center}
    }
}

\setlength{\droptitle}{-2em}

  \title{\large{Mortality Estimates for Small Areas in Argentina (2009-2011)}\footnote{Este
  paper fue exclusivamente elaborado para presentar en el 8° Congreso
  Internacional de ALAP realizado en Ciudad de Puebla, México en Octubre
  de 2018. Por favor, no reproducir o citar sin el permiso expreso de
  los autores.} \vspace{0.5cm}}
    \pretitle{\vspace{\droptitle}\centering\huge}
  \posttitle{\par}
    \author{\normalsize \emph{Nicolás Sacco}\footnote{Cedeplar-UFMG,
  \href{mailto:nsacco@cedeplar.ufmg.br}{\nolinkurl{nsacco@cedeplar.ufmg.br}}};
\emph{Iván Williams}\footnote{Universidad Nacional de Luján,
  \href{mailto:ivanwilliams1985@gmail.com}{\nolinkurl{ivanwilliams1985@gmail.com}}};
\emph{Bernardo L. Queiroz}\footnote{Cedeplar-UFMG,
  \href{mailto:lanza@cedeplar.ufmg.br}{\nolinkurl{lanza@cedeplar.ufmg.br}}}}
    \preauthor{\centering\large\emph}
  \postauthor{\par}
    \date{}
    \predate{}\postdate{}
  
%% Configurado según normas editoriales de ALAP: 
%% www.alapop.org/alap/ComitePublicaciones/MANUAL_DE_NORMAS_EDITORIALES_ALAP.pdf


%% FONT
\usepackage[T1]{fontenc}
\usepackage[utf8]{inputenc}
\usepackage{mathptmx} 

%% LENGUAGE
\usepackage{polyglossia}
\setdefaultlanguage[]{spanish}
\setmainlanguage[]{spanish}
\setotherlanguage[]{english}


%% PAGE DIMENSIONS
\usepackage{geometry}
  \geometry{
  letterpaper,
  left=3cm,
  right=3cm,
  top=2.54cm,
  bottom=2.54cm
  }
 
%% SPACING
\usepackage{setspace}
\spacing{1.0}

%% LINE AND PAGE BREAKING
%\sloppy
%\clubpenalty = 10000
%\widowpenalty = 10000
%\brokenpenalty = 10000
%\usepackage{microtype}

%% PARAGRAPH BREAKS/ INDENT
\setlength{\parskip}{0pt}
\setlength{\parindent}{1.25cm}

% Notas al pie de página 
\footnotesize\fontsize{10pt}{12pt}


%% Títulos
\usepackage{titlesec}
\titleformat*{\section}{\normalfont\normalsize\bfseries}
\titleformat*{\subsection}{\normalfont\small\bfseries}
\titleformat*{\subsubsection}{\normalfont\small\bfseries\itshape}
\titleformat{\paragraph}{\normalfont\small\itshape}{\theparagraph}{1em}{}
\titlespacing*{\paragraph}{0pt}{3.25ex plus 1ex minus .2ex}{1.5ex plus .2ex}


%% MATHS
%\usepackage{bm,amssymb,amsmath}
%\allowdisplaybreaks


%% CAPTIONS
%\usepackage{caption}
%\DeclareCaptionStyle{italic}[justification=centering]
% {labelfont={bf},textfont={it},labelsep=colon}
%\captionsetup[figure]{style=italic,format=hang,singlelinecheck=true}
%\captionsetup[table]{style=italic,format=hang,singlelinecheck=true}

%% GRAPHICS
\usepackage{graphicx}
\usepackage{grffile}

%% HYPERLINKS
\usepackage{xcolor} % Needed for links
\definecolor{darkblue}{rgb}{0,0,.6}
\usepackage{url}

\usepackage[]{hyperref}
\hypersetup{
      colorlinks=true,
      linkcolor=red,
      linktoc=all,
      urlcolor=darkblue,
      hyperfootnotes=true,
      breaklinks=false,
      pagebackref=true,
      hyperindex=true,
      citecolor=blue,
      filecolor=orange,
      }


% CONFIGURACIÓN DEL PDF
\hypersetup{
     pdftoolbar=true,
     pdfmenubar=true,
     pdfnewwindow=true,
     pdfsubject={Mortalidad subnacional en Argentina},
     pdfkeywords={Mortalidad; Áreas menores; Argentina},
     pdfauthor={Nicolás Sacco},
     pdftitle={Mortality Estimates for Small Areas in Argentina (1980-2015)},
     pdfproducer={Bookdown with LaTeX}
}


\usepackage[nottoc,notlot,notlof]{tocbibind}
\usepackage{tocloft}



%%  ACRÓNIMOS Y A ABREVIATURAS
%%[automake,abbreviations,nomain,acronym,xindy,toc,style=indexgroup]
%%\usepackage{glossaries}
\usepackage[automake,abbreviations,nomain,acronym,xindy,toc]{glossaries-extra}
\setabbreviationstyle[acronym]{long-short}
\makeglossaries
\usepackage[xindy]{imakeidx}
\makeindex
%\usepackage{nomencl}
%\makenomenclature
% Acr'onimos

\newacronym{ibge}{IBGE}{Brazilian Institute of Geography and Statistics}
\newacronym{deis}{DEIS}{Dirección de Estadísticas e Información de Salud}
\newacronym{indec}{INDEC}{Instituto Nacional de Estadística y Censos}
\newacronym{oms}{OMS}{Organización Mundial de la Salud}
\newacronym{tfr}{TFR}{Total Fertility Rate}
\newacronym{un}{UN}{United Nations}
\newacronym{celade}{CELADE}{Latin American and Caribbean Demographic Centre}
\newacronym{wpp}{WPP}{World Population Prospects} 
\newacronym{alyc}{ALyC}{América Latina y el Caribe}
\newacronym{alap}{ALAP}{Asociación Latinoamericana de Población}
\newacronym{tbm}{TBM}{Tasa Bruta de Mortalidad}
\newacronym{ddm}{DDM}{Death Distribution Methods} 

%% Abreviaturas

\newabbreviation{evn}{EVN}{esperanza de vida al nacer}




%\usepackage{babel}
\usepackage{csquotes}
%\usepackage{keyval}
%\usepackage{ifthen}

%\usepackage{xpatch}
%\usepackage{showkeys}


%\usepackage[square,numbers]{natbib}

%\bibliography{bib/refsubmort.bib}

%\usepackage[style=authoryear-comp, backend=biber, natbib=true]{biblatex}

%%\ExecuteBibliographyOptions{style=authoryear-comp, bibencoding=utf8,minnames=1,maxnames=3, maxbibnames=99,dashed=false,terseinits=true,giveninits=true,uniquename=false,uniquelist=false,doi=false, isbn=false,url=true,sortcites=false}






%%\usepackage[backend=biber, style=alphabetic, citestyle=authoryear, natbib=true]{biblatex}

\begin{document}
\maketitle

%%% Doc-Prefix


\pagenumbering{roman}

\renewcommand{\abstractname}{Resumen}

\begin{abstract}
\setlength{\parindent}{2pt}
\noindent
En América Latina y el Caribe la demanda de datos a nivel sub-nacional es creciente, tanto como herramienta para la aplicación de distintos planes de desarrollo como para la asignación de recursos. Estudios recientes en los países desarrollados apuntan a señalar que las estimaciones de mortalidad en áreas menores tienden a ser diferenciales, encontrándose contrastes en la esperanza de vida al nacer entre distintas sub-regiones y/o grupos sociales, idea que discrepa, en parte, con la observada declinación en la variabilidad de la edad a la defunción desde mediados del siglo XX en países desarrollados, que sugiere que la mayoría de las muertes se concentran en una edad cada vez más estrecha y a la vez, más estable a medida que la mortalidad desciende. Considerando que no hay antecedentes en la temática para el caso argentino, en artículo se propone realizar estimaciones de estructura y niveles mortalidad para áreas pequeñas (sub-provinciales -departamentos-) en Argentina (provincias seleccionadas) durante el periodo 1980-2015. Para ello se propone combinar métodos indirectos de estimación demográfica y estadísticos. El plan de trabajo permitirá construir un insumo tanto para la elaboración de políticas públicas como para profundizaren los diferenciales de la mortalidad en las zonas geográficas seleccionadas. 
\\\\
Palabras clave: Mortalidad; Argentina; Áreas menores.
\end{abstract}


\renewcommand{\abstractname}{Abstract}
\noindent
\begin{abstract}
\noindent
En América Latina y el Caribe la demanda de datos a nivel sub-nacional es creciente, tanto como herramienta para la aplicación de distintos planes de desarrollo como para la asignación de recursos. 
Estudios recientes en los países desarrollados apuntan a señalar que las estimaciones de mortalidad en áreas menores tienden a ser diferenciales, encontrándose contrastes en la esperanza de vida al nacer entre distintas sub-regiones y/o grupos sociales, idea que discrepa, en parte, con la observada declinación en la variabilidad de la edad a la defunción desde mediados del siglo XX en países desarrollados, que sugiere que la mayoría de las muertes se concentran en una edad cada vez más estrecha y a la vez, más estable a medida que la mortalidad desciende. 
Considerando que no hay antecedentes en la temática para el caso argentino, en artículo se propone realizar estimaciones de estructura y niveles mortalidad para áreas pequeñas (sub-provinciales -departamentos-) en Argentina (provincias seleccionadas) durante el periodo 1980-2015. Para ello se propone combinar métodos indirectos de estimación demográfica y estadísticos. El plan de trabajo permitirá construir un insumo tanto para la elaboración de políticas públicas como para profundizaren los diferenciales de la mortalidad en las zonas geográficas seleccionadas.
\\\\
Keywords: Mortalidad; Argentina; Áreas menores.
\end{abstract}



{
\setcounter{tocdepth}{5}
\tableofcontents
}

\clearpage\pagenumbering{arabic}\setcounter{page}{1}

\hypertarget{sec:intro}{%
\section{Introducción}\label{sec:intro}}

Estudios recientes en los países desarrollados apuntan a señalar que las
estimaciones de mortalidad en áreas menores tienden a ser diferenciales,
encontrándose contrastes en la esperanza de vida al nacer entre
distintas sub-regiones y/o grupos sociales (Chetty et~al.
\protect\hyperlink{ref-ChettyEtAl2016}{2016}), idea que discrepa, en
parte, con la observada declinación en la variabilidad de la edad a la
defunción desde mediados del siglo XX en países desarrollados, que
sugiere que la mayoría de las muertes se concentran en una edad cada vez
más estrecha y a la vez, más estable a medida que la mortalidad
desciende (Wilmoth y Horiuchi
\protect\hyperlink{ref-WilmothHoriuchi1999}{1999}). Estos trabajos han
propiciado el debate reciente entre investigadores y políticas públicas
acerca de las raíces de las causas de la desigualdad en la mortalidad y
cómo dar cuenta de ella.

A pesar del amplio trabajo en países desarrollados al respecto, es poco
lo que se conoce sobre las diferencias a nivel sub-nacional en la
mortalidad (adulta e infantil) en el Cono Sur. En particular, en
Argentina, las estimaciones de mortalidad y el conocimiento de sus
niveles y tendencias, están limitados, como en la mayoría de los países
de la región, por la calidad y disponibilidad de los datos. Los
problemas más frecuentes en este país están asociados a la cobertura
incompleta de los datos del sistema de registros vitales, errores en la
declaración de la edad en las poblaciones involucradas en los cálculos
de las tasas (población y muertes) y la falta de información sobre
causas de muerte.

Los cálculos para estimar la estructura y niveles de mortalidad son
fuertemente dependientes de la disponibilidad de esos datos y del
detalle de la información. Para los casos en los cuales las estadísticas
vitales son completas, la mortalidad puede medirse directamente desde
esa fuente. Desafortunadamente, la mayoría de los países de \gls{alyc},
sobre todo para el periodo anterior a 1950, no poseen datos de registros
de vitales de confianza o simplemente se carece de ellos. A medida que
mejora la calidad del sistema de registro, estadísticas vitales más
completas deberían poder ser obtenidas para áreas administrativas
pequeñas (Setel et~al. \protect\hyperlink{ref-SetelEtAl2007}{2007}).

Para la mortalidad, estudios recientes realizados enfocados en Brasil
(Lima, Queiroz, y Sawyer
\protect\hyperlink{ref-LimaQueirozSawyer2014}{2014}; Lima y Queiroz
\protect\hyperlink{ref-LimaQueiroz2014}{2014}; Freire
\protect\hyperlink{ref-FreireEtAl2015}{2015}; Queiroz et~al.
\protect\hyperlink{ref-QueirozFreireGonzagaEtAl2017}{2017}) han aplicado
distintas metodologías para estimar la cobertura del sistema de registro
de muertes en el horizonte sub-nacional, tanto por su interés \emph{per
se} como por su utilidad como insumo para el cálculo de los niveles y
estructura de la mortalidad en áreas administrativas menores. Hay
razones para sospechar que lo observado en Brasil puede darse también en
Argentina, a pesar de que ambos países transcurrieron por muy distintos
procesos de transición epidemiológica.

\hypertarget{sec:obj}{%
\subsection{Objetivos}\label{sec:obj}}

Considerando que no hay antecedentes en la temática para el caso
argentino, se plantea realizar estimaciones de estructura y niveles
mortalidad para áreas pequeñas (sub-provinciales -departamentos-) en
Argentina durante el periodo 1980-2015. Se propone para ello combinar la
aplicación de métodos demográficos \gls{ddm} de estimación indirecta y
estadísticos (métodos Bayesianos), que serán más adelante explicitados
en la Sección \ref{sec:met}.

Los datos sobre defunciones y causas de muerte son esenciales para
establecer prioridades de inversión en servicios públicos, la aplicación
de políticas de planificación económica y su monitoreo, a nivel nacional
y, cada vez con mayor intensidad, a niveles administrativos menores
(Setel et~al. \protect\hyperlink{ref-SetelEtAl2007}{2007}). En
\gls{alyc} este tipo de información, a nivel local, es escaso o bien
inexistente --incluso teniendo en cuenta encuestas ad-hoc-- y la demanda
por esos datos es creciente, tanto como insumo para la aplicación de
distintos planes de desarrollo como para la asignación de recursos.

Adicionalmente, si bien se trata de un debate aún abierto en cuanto a su
nivel (Grushka \protect\hyperlink{ref-Grushka2010}{2010}), dado que en
el futuro se espera que la \gls{evn} continúe en aumento (Rofman
\protect\hyperlink{ref-Rofman2007}{2007}), las transformaciones sociales
producidas por los cambios en la mortalidad ofrecen a las próximas
décadas del siglo XXI un horizonte completamente distinto de aquel
observado a fines del siglo XIX y a lo largo del siglo XX. Estos
procesos tornan necesario comparar la experiencia de diferentes unidades
sub-nacionales que puedan brindar ideas adicionales sobre la magnitud de
los cambios en la mortalidad a lo largo de los años y sus vínculos con
el desarrollo económico.

En los tiempos que corren, el monitoreo de metas y políticas de salud
demanda no sólo contar con estadísticas vitales a niveles de total país
sino también a menores niveles de desagregación. Es que los promedios
generales no suelen incorporar la particularidad de coyunturas locales,
aspecto que los diferenciales geográficos pueden llegar a mostrar. Esta
situación, a la vez, afecta la posibilidad de construir perfiles
epidemiológicos particulares a niveles de observación locales,
posibilitando un diagnóstico incorrecto de aplicación de políticas
públicas que en definitiva impacten de forma no esperada en la salud de
la población.

\hypertarget{ch:litreview}{%
\section{Literature Review}\label{ch:litreview}}

Es bien conocido y ha sido documentado en diversos estudios el proceso
por el cual en la Argentina las tasas de mortalidad se redujeron
sustancialmente, aunque no de forma constante, merma que se debió a un
desarrollo socioeconómico precoz en relación con el resto de \gls{alyc},
al alto grado de urbanización del país y a la expansión de la educación
formal (Recchini de Lattes y Lattes
\protect\hyperlink{ref-RecchinideLattes1975}{1975}). Estos elementos se
conjugaron en un contexto mundial de crecimiento histórico de la
\gls{evn} (Oeppen y Vaupel
\protect\hyperlink{ref-OeppenVaupel2002}{2002}; Riley
\protect\hyperlink{ref-Riley2005}{2005}) y de continuas reducciones de
la mortalidad a edades avanzadas en muchos países de ingresos altos y
medios (Rau et~al. \protect\hyperlink{ref-RauEtAl2008}{2008}), aunque al
mismo tiempo se observaron patrones emergentes de una creciente brecha
en la longevidad entre regiones con diferente grado de desarrollo
relativo (Cohen, Preston, y Crimmins
\protect\hyperlink{ref-CohenPrestonCrimmins2011}{2011}; Meslé y Vallin
\protect\hyperlink{ref-MesleVallin2011}{2011}).

Durante los últimos 130 años, las tasas de mortalidad en Argentina se
han reducido sustancialmente, aunque no de una forma simple o lineal.
Sus fluctuaciones se han visto afectadas por una serie de
acontecimientos históricos y sociales, que han dejado su huella en la
estructura de la población. Las series disponibles de esta medida para
la Argentina a nivel del total del país, comienzan alrededor del año
1870. La \gls{tbm} fue estable, alrededor del 30 por mil desde esa fecha
hasta fines del siglo XX. Luego de ese momento hay una marcada y
continua declinación que llega a valores menores al 10 por mil,
alrededor de mediados del siglo XX. Desde ese periodo y hasta la primera
década del siglo XXI, la \gls{tbm} se encuentra prácticamente detenida,
descendiendo lentamente a un nivel cercano al 8 por mil. Este
estancamiento en la caída de la \gls{tbm} es parte de un proceso de
desaceleración y al mismo tiempo fruto del proceso del continuo ascenso
de la esperanza de vida, compensado por el 'envejecimiento de la
población (Grushka \protect\hyperlink{ref-Grushka2014}{2014}).

Durante un largo período la Argentina no contó con información completa
sobre muertes ocurridas por año clasificadas por edad. Recién a partir
de 1911 fueron publicadas las primeras y rudimentarias series
estadísticas sobre muertes, y con la realización del tercer Censo
Nacional, en 1914, se dispuso por primera vez de datos de muertes y de
población (Somoza \protect\hyperlink{ref-Somoza1973}{1973}; Recchini de
Lattes y Lattes
\protect\hyperlink{ref-RecchinideLattes1975}{1975})\footnote{Información
  que todavía adolece de distintos tipos de errores, véase por ejemplo
  (Mazzeo \protect\hyperlink{ref-Mazzeo2005}{2005}). En su evaluación
  sobre las estadísticas vitales en \gls{alyc}, Jaspers y Orellana
  (\protect\hyperlink{ref-JaspersOrellana1994}{1994}) observaron que en
  gran parte de los países de la región el subregistro más alto se da
  entre el grupo de 0 a 14 y también en el de los 50 a 70 años
  aproximadamente; en las edades más avanzadas (75 años y más) observó
  un sobrerregistro de las defunciones (que se explica, en parte, por la
  exageración en la declaración de la edad de las personas fallecidas de
  mayor edad), tendencias se observan más claramente entre los varones
  que entre las mujeres}.

Con la información derivada de registros de muertes de nueve provincias
y la Ciudad de Buenos Aires (cubriendo aproximadamente al 80\% de la
población total del país), Somoza y equipo
(\protect\hyperlink{ref-Somoza1968}{1968},
\protect\hyperlink{ref-Somoza1970}{1970},
\protect\hyperlink{ref-Somoza1971}{1971},
\protect\hyperlink{ref-Somoza1972}{1972},
\protect\hyperlink{ref-Somoza1973}{1973}) construyeron, a fines de la
década del sesenta, las tablas de vida transversales para los períodos
de 1913-1915, 1946-1948 y 1959-1961, es decir, en torno a las fechas en
las cuales se realizaron los censos de 1914, 1947 y 1960. Además,
utilizando métodos de estimación demográfica, consistentes en la
comparación de información de dos censos sucesivos, obtuvieron
aproximaciones a los niveles y a la estructura de la mortalidad para
momentos previos a 1914. De esa forma se pudo conocer, aproximadamente,
la esperanza de vida en los períodos intercensales de 1869-1895 y de
1895-1914. Si bien el esfuerzo de Somoza y equipo no fue el primero ni
el único\footnote{Para mayor detalle véase Müller (1973)}, estas tablas
forman el insumo que podemos llamar \textgreater{}oficial\textgreater{}
de la historia de la mortalidad en la Argentina, que luego adoptarían
los organismos nacionales e internacionales como válidos y para realizar
proyecciones, retroproyecciones y estimaciones de población.

Luego de las de Somoza, Müller
(\protect\hyperlink{ref-Mueller1978}{1978}) elaboró las tablas
transversales de mortalidad para el período 1969-1971. Llamativamente
por primera vez la \gls{evn} bajaba en la Argentina. Esta tabla fue
construida para el total del país y cuatro regiones, con una metodología
similar a las de Somoza. Después de ese lapso ya el \gls{indec} se hizo
cargo oficialmente de calcular y publicar las estadísticas oficiales de
la esperanza de vida.

Desde 1914, si bien las mejoras no fueron uniformes durante todo el
período, se observan pocas desviaciones de una tendencia lineal. Luego
de moderado a comienzos del siglo XX, se distingue un crecimiento
constante, pero de forma más lenta que en el pasado. La \gls{evn}
continuó aumentando en forma sostenida hasta la década de 1960. De
acuerdo al dato oficial este comportamiento regular y de continuo
crecimiento de la \gls{indec}, en ambos sexos, se interrumpe hacia 1970
con un aparente retroceso, de 1,8 años, respecto del decenio anterior.
Aún no ha sido lo suficientemente bien abordado este fenómeno. Müller y
Accinelli (\protect\hyperlink{ref-MuellerAccinelli1980}{1980}) lo
atribuyeron a un supuesto arribo a un umbral en las ganancias de
mortalidad impuesto por las condiciones socioeconómicas del país. Esta
hipótesis sugirió que los progresos sanitarios perdieron su
independencia de los niveles de desarrollo económico.

Esta interpretación se sustentaba en la idea de que los principales
techos alcanzados en los valores de \gls{evn} se daban en las regiones
con mayor desarrollo socioeconómico de la Argentina, mientras que, en el
resto de las zonas, de menor desarrollo relativo, mostraban un ascenso
en la \gls{evn}. En efecto, en la tabla de mortalidad de 1969-1971
mostró un aumento de la mortalidad que afectó especialmente al sexo
masculino, observándose que la \gls{evn} de los varones desciende
respecto de la registrada para el periodo 1959-1961 en las regiones de
mayor desarrollo relativo del país, mientras que aumenta en regiones de
menor desarrollo, donde el nivel de la \gls{evn} era menor en el periodo
anterior. Para las mujeres, en cambio, la \gls{evn} pareció mostrarse
prácticamente estable al nivel del total del país, mientras que
disminuyó en la región Buenos Aires, pero de forma más moderada que para
los varones.

Si bien falta aún estudio exhaustivo de lo que sucedió durante ese
lapso, se han formulado pocas hipótesis alternativas, aún no exploradas
del todo. Una de ellas sugiere que una posible explicación a este
comportamiento, incoherente, de la esperanza de vida --en un contexto de
amplio desarrollo socioeconómico-- podría deberse al hecho del posible
aumento en la cobertura de defunciones y en la calidad de certificación
de causas de muerte, gracias a la formalización del Sistema de
Estadísticas Vitales en el año 1968 (Bankirer
\protect\hyperlink{ref-Bankirer2010}{2010}). Sin embargo, para comprobar
tal hipótesis deberían revisarse las estadísticas vitales de ese
período, ya que se supone que los que elaboraron la tabla conocían las
limitaciones de los datos construidos.

Aunque moderados si se los compara con los años previos, los avances en
la \gls{evn} que ocurrirían en el período posterior, a pesar del
deterioro de las condiciones socioeconómicas de la Argentina en su
conjunto hasta el año 2003, mostrarían los límites de tal hipótesis. Es
que ya a partir de 1980 la \gls{evn} comienza nuevamente a mostrar una
tendencia coherente con el descenso de la mortalidad. Este período fue
seguido de una recuperación de la tendencia ascendente, pero de forma
más lenta, alcanzado para el 2009 una \gls{evn} para ambos sexos de
75,34 años. Este comportamiento es consistente con el esperado aumento
de la \gls{evn} cuando decrecen los niveles de mortalidad, siguiendo el
comportamiento que distintos autores consideran como el de una curva
logística.

\hypertarget{sec:hip}{%
\subsection{Hipótesis}\label{sec:hip}}

Como marco general se puede contextualizar la Argentina, a partir de su
modernización y exitosa integración con el mercado mundial a fines del
siglo XIX, como un país anómalo en términos demográficos y sanitarios:
receptor importante de inmigrantes con una amplia cobertura del sistema
educativo, y donde el bajo desempleo, el bajo costo de los alimentos y
el desarrollo de un saneamiento eficaz en las ciudades producía un buen
nivel sanitario, y consecuentemente una baja tasa de mortalidad,
infantil y adulta. Sin embargo, este camino -pionero a nivel mundial- se
desacelera a partir de mediados de la década de 1970.

Bajo este marco, este trabajo proponer contrastar empíricamente el
vínculo, a nivel desagregado, entre los niveles y estructura de la
mortalidad a nivel sub-nacional que las aproximaciones teóricas ponderan
en distinta medida como vinculadas, a largo plazo, con desarrollo
económico, a partir de la década de 1980.

¿Qué rol juega la dinámica de la mortalidad sub-nacional en la
configuración de los niveles de mortalidad a nivel regional y total
país? ¿Se debe la desaceleración del descenso de la mortalidad en la
Argentina a la existencia de sectores socialmente heterogéneos,
\textgreater{}rezagados\textgreater{} en el proceso de transición
epidemiológica? ¿Se asocian patrones diferenciales a distintos modelos
desarrollo económico aplicados a lo largo de distintos ciclos
socio-históricos, a partir de la década de 1980? Hipótesis principal:
los niveles de mortalidad y el ritmo de su descenso, a nivel provincial
y total país, son dependientes de la creciente desigualdad social,
aspecto posible de vislumbrar mediante la observación sub-nacional de la
mortalidad.

En base a estas preguntas e hipótesis de trabajo se propone estudiar los
niveles y la estructura de la mortalidad a nivel sub-nacional
(departamentos) en lo relativo a dos dimensiones concretas:

\begin{enumerate}
\def\labelenumi{\alph{enumi})}
\tightlist
\item
  como elemento de la evolución a largo plazo del desarrollo social y
  económico ,
\item
  como factor asociado a distintas etapas de la transición demográfica y
  epidemiológica.
\end{enumerate}

\hypertarget{sec:dat}{%
\section{Data and Methods}\label{sec:dat}}

Incluido en este reporte/repositorio se encuentran las bases de datos
originales llamadas \texttt{DEFT2009.DBF}, \texttt{DEFT2010.DBF},
\texttt{DEFT2011.DBF}, \texttt{DEFT2012.DBF} y \texttt{DEFT2013.DBF}.
Contienen información anual sobre Defunciones oficiales anuales de la
\gls{deis} según distintas variables, para Argentina, a nivel de total
país. El diseño de registro de estas bases puede consultarse en el
archivo
\href{../analysis/data/deis/COD_DEF2001_2015.xls}{COD\_DEF2001\_2015},
junto con los códigos por jurisdicción
\href{../analysis/data/deis/cods_prov_depto_cen010.xls}{cods\_prov\_depto\_cen010}.

De estas bases se toman las variables \enquote{Departamento de
residencia}, \enquote{Jurisdicción de residencia}, \enquote{Edad},
\enquote{Unidad de edad}, \enquote{Sexo} y \enquote{Fecha de defunción}.
Sus variables fueron ajustadas y ordenadas para generar una base de
datos limpia y ordenada, con las mismas definiciones de categorías en
cada variable.

\begin{verbatim}
##   DEPRE PROVRE SEXO EDAD_r anio fd_aa fd_mm ndef
## 1     1      2    1      0 2013  2012    12    1
## 2     1      2    1      0 2013  2013     1    1
## 3     1      2    1      0 2013  2013     3    1
## 4     1      2    1      0 2013  2013     4    1
## 5     1      2    1      0 2013  2013     5    2
## 6     1      2    1      0 2013  2013     6    2
\end{verbatim}

\begin{verbatim}
##       
##          2008   2009   2010   2011   2012   2013
##   2009   1838 174062      0      0      0      0
##   2010      0   1576 177771      0      0      0
##   2011      0      0   2210 173363      0      0
##   2012      0      0      0   2287 174465      0
##   2013      0      0      0      0   2620 177090
\end{verbatim}

Fue utilizada la información proveniente del registro de hechos vitales
(1980-2015), elaborados por la \gls{deis} -dependiente del Ministerio de
Salud de la Nación-, que produce estadísticas anuales correspondientes
al total de registros de nacimientos, defunciones, defunciones fetales y
matrimonios, con cobertura territorial nacional.

Se tomaron las bases originales y fueron transformadas

El diseño de registro de la base de datos (\emph{raw data}) fue el
provisto por la \gls{deis} y se trata de este archivo:

\hypertarget{la-calidad-de-los-datos}{%
\subsection{La calidad de los datos}\label{la-calidad-de-los-datos}}

Dentro del contexto latinoamericano\footnote{De acuerdo al \gls{celade}
  (\protect\hyperlink{ref-CELADE2012}{2012}) ha mejorado mucho la
  calidad de las estadísticas vitales en la región desde mediados del
  siglo XX: en lo que respeta a la cobertura, se estima que la región
  pasó de un sub-registro promedio del 75\% a 22\%. Sin embargo, la
  persistencia de algunos desafíosse mantiene, siendo la cobertura a
  nivel sub-nacional, un aspecto muy importante (Ribotta
  \protect\hyperlink{ref-Ribotta2016}{2016})}, Argentina se ha
caracterizado por presentar una buena cobertura de registro de
vitales.\footnote{Algunas evaluaciones parciales respecto de la calidad
  de información de este sistema -que previo a la década de 1990
  conservaba grandes inconvenientes de variabilidad e integridad de sus
  registros- sugirieron resultados a veces no siempre compatibles
  (Torrado \protect\hyperlink{ref-Torrado1993}{1993}, 307). Pantelides
  (\protect\hyperlink{ref-Pantelides1989}{1989}, 9--10) aludió a una
  serie de graves problemas de calidad (tales como de definiciones,
  criterios de registro, omisiones y otros) a lo largo de la serie, que
  fueron de a poco mejorándose entre 1980 y 1990. Si bien se especulaba
  que ya para fines del siglo XX el sistema de registro había mejorado
  con respecto al pasado y era, en términos generales, aceptable, un
  estudio específico (INDEC-UNICEF
  \protect\hyperlink{ref-INDEC-UNICEF2003}{2003}) reveló una
  significativa omisión. Hacia el año 2001 los formularios se
  rediseñaron y gradualmente pareció haber mejorado la calidad de la
  información registrada (Población
  \protect\hyperlink{ref-Poblacion2013}{2013}, 23--27).} Los estudios
más recientes que estimaron los niveles de cobertura de los nacimientos
y las defunciones infantiles para Argentina (Fernández et~al.
\protect\hyperlink{ref-FernandezEtAl2008}{2008}), muestran mejoras en el
sistema de estadísticas vitales de Argentina a lo largo del tiempo, que
no suele ser la norma en el resto de los países de la región. Si bien se
han aplicado distintas metodologías directas para estimar la cobertura
en distintos contextos sub-nacionales --véase al respecto Ribotta
(\protect\hyperlink{ref-Ribotta2016}{2016})--, en el caso particular de
Argentina, sólo se cuenta con un estudio que investigó los niveles de
cobertura en lo que respecta exclusivamente a los nacimientos\footnote{Salvo
  el trabajo de Sacco
  (\protect\hyperlink{ref-Sacco2016a}{2016}\protect\hyperlink{ref-Sacco2016a}{b}),
  que, utilizando \gls{ddm}, analizó los datos de mortalidad para el
  periodo 2009-2011--, a nivel del total del país}.

Esa investigación mostró que los datos censales informaban un mayor
número de nacimientos que los datos provenientes de los registros, pero,
comparando ambas fuentes, concluía que la omisión de registro nacidos
vivos era de calidad aceptable (Fernández et~al.
\protect\hyperlink{ref-FernandezEtAl2008}{2008}, 125). El informe de la
\gls{oms} (\protect\hyperlink{ref-OMS2012}{2012}) ubicó a la Argentina,
a nivel global, como uno de los países con mejores niveles de cobertura
del registro de nacimiento, mayores al 90\%.

\% de desconocidos por sexo. Por ejemplo, Buenos Aires\ldots{}tiene 26\%
de desconocidos en algunos departamentos por eso no discriminamos por
sexo. Agregar gráfico de desconocidos por sexo en cada provincia.

Gráfico por edad. Todas las provincias tienen deptos por abajo del 2\%
de desconocidos en edad.

Agregar tabla.

El análisis será focalizado luego de la segunda mitad del siglo XX, a
partir de la década de 1980, por razones del corpus y por la escasez de
datos y por su baja calidad previo a esa fecha, debido al sub-registro
estructural y/o deliberado de las tasas de mortalidad en Argentina.

\hypertarget{procedimientos-metodologicos}{%
\subsection{Procedimientos
Metodológicos}\label{procedimientos-metodologicos}}

Los procedimientos metodológicos propuestos para cumplir con los
objetivos planteados, siguiendo a Freire et. al.
(\protect\hyperlink{ref-FreireEtAl2015}{2015}) remiten a aplicar
\gls{ddm} (Timæus, Dorrington, y Hill
\protect\hyperlink{ref-TimaeusDorringtonHill2013}{2013}), en primer
lugar, y luego realizar estimaciones bayesianas a nivel sub-nacional.
Específicamente se propone las siguientes actividades y objetivos
particulares:

Para el objetivo 1),

\begin{enumerate}
\def\labelenumi{\alph{enumi}.}
\tightlist
\item
  Evaluar la calidad de los datos de población (denominador) y
  defunciones (numerador) por edades, por provincia y departamentos.
\item
  Evaluar y diagnosticar la necesidad de ajuste de numerador y
  denominador.
\item
  Estimar tasas de mortalidad por edad y evaluar la necesidad de
  tratamiento estadístico.
\end{enumerate}

Para el objetivo 2):

\begin{enumerate}
\def\labelenumi{\alph{enumi}.}
\setcounter{enumi}{3}
\tightlist
\item
  Seleccionar métodos de estimación por departamento.
\item
  Estimar intervalos de confianza.
\item
  Elaborar tablas de mortalidad y esperanzas de vida.
\item
  Analizar heterogeneidad entre departamentos.
\end{enumerate}

Objetivos específicos: FASE 1 Elección de provincias Caracterizar la
transición de cada provincia (brevísimo) Corrección de cobertura a nivel
subprovincial: poblacion (denominador) y defunciones (numerador) por
edades Para población se puede tomar los datos INDEC 2010 conciliados, y
aplicarle estructura observada (pero corregida) de Censo Para
defunciones se puede corregir mediante DDM (Métodos de distribución de
muertes) Estimación de las tasas de mortalidad por edad a nivel
subprovincial: diagnóstico y necesidad de estimación estadística FASE 2
Selección de métodos de estimación por departamento: búsqueda
bibliográfica breve Estimación e intervalos de confianza\\
Cómo medir heterogeneidad interna

Si bien se suele catalogar a Argentina como un país con buenas
estadísticas de muerte (fuente), de lo que se conoce existe un
porcentaje desigual de defunciones infantiles no registradas por
provincia (fuente DEIS). En este sentido, como la fuente de datos es un
registro, pensando en la estimación de tasas de mortalidad por edad,
podríamos pensar que la varianza sería nula y el sesgo (ambos
componentes del error medio cuadrático de un estimador) vendría dado por
el patrón de casos omitidos en cada jurisdicción. El segundo de los
componentes del error no será tratado en esta instancia de la
investigación, aunque pueden verse intentos crecientes por abordar el
tema, dependiendo de la información auxiliar con la que se cuente
(Schmertmann y Gonzaga (2018), Queiroz y otros (2017)) (tampoco el que
refiere a los problemas de conteo censal). Respecto al primero, no
podemos pensarla nula debido a que existen fenómenos con una cantidad
pequeña de \enquote{experimentos} (pocos expuestos en nuestro caso), que
presentan una mayor varianza en sus estimaciones, por lo que requiere un
tratamiento especial con el fin de reflejar el riesgo de mortalidad
subyacente. Con este fin utilizaremos el método bayesiano empírico, en
post de mejorar la eficiencia estadística de los estimadores de tasas de
mortalidad, disminuyendo la varianza en los casos de jurisdicciones
pequeñas (Efron y Morris (1975), Marshall (1990), Longford (2005),
Assuncao (2005)). La distribución a priori corresponde a la distribución
conjunta del vector de tasas de mortalidad por edad del área mayor.
Luego, mediante lo observado en cada área menor, se produce el ajuste
bayesiano de la distribución de mortalidad a posteriori. La
característica de \enquote{empírico} radica en que las distribuciones de
los parámetros del área mayor son estimadas a partir de los datos. La
definición de área mayor debe explotar la idea de similitud interna,
para poder suponer que la mortalidad de áreas menores son realizaciones
de un proceso estocástico mayor, supuesto importante. La similitud en
los patrones de mortalidad debe aproximarse de manera indirecta, lo que
en nuestro caso, y por ser una investigación incipiente, se supone por
pertenencia a la misma provincia, cuestión que no siempre es acertada,
donde la \enquote{distancia} entre jurisdicciones no se mide por
kilómetros (Longford, 2005). Considerando un grupo de edad quinquenal
cualquiera en un área i, la distribución de defunciones d se supone un
proceso de Poisson, donde E(dx,4i/mx,4i) =Nx,4imx,4i , siendo N los
expuestos y m la tasa de mortalidad. Siendo mx,4i el estimador de máxima
verosimilitud Dx,4i/Nx,4i , y dada una distribución a priori del
parámetro mx,4 , entonces su esperanza no condicionada sería
Em\{E(mx,4i/mx,4i )\}=Em\{mx,4i\}=mx,4 , y su varianza no condicionada
Vm(E(mx,4i/mx,4i ))+Em(V(mx,4i/mx,4i ))
=Vm(mx,4i)+Em(mx,4iNx,4i)=Vm(mx,4i)+mx,4Nx,4i. El estimador lineal
bayesiano mx,4i que minimiza el error cuadrático medio de mx,4i (e
indicadores que sean funciones lineales de este) es (Robbins, 1983):
mx,4i=mx,4i+Sx,4i(mx,4i-mx,4i) De nuevo, es empírico porque mx,4 se
reemplaza por mx,4 , la media ponderada de las áreas menores. El factor
Sx,4i (de contracción o \enquote{shrinkage}) es el ratio entre la
esperanza de la varianza muestral del estimador en el área menor y la
varianza no condicionada del estimador, lo que termina siendo:
Sx,4i=Vmmx,4iVmmx,4i+mx,4Nx,4i O visto de otra forma, el ratio entre la
varianza del área menor respecto a la suma de la varianza total (del
área menor y mayor), en sintonía con un análisis de la varianza clásico
(ANOVA). Siguiendo este razonamiento, en un contexto de extrema
homogeneidad, se podría caracterizar un área menor muy pequeña a partir
de la estimación del área mayor (Sx,4i≅1). Por otro lado, áreas de alto
peso poblacional tomarán valores cercanos a los observados (Sx,4i≅0). En
el medio de estos extremos, la función combina linealmente la estimación
del área mayor respecto a la del área menor, siendo la ponderación entre
ambas tal que las jurisdicciones con población pequeña (el caso extremo
sería aquella que no presente eventos en alguna categoría etaria), se
acercarán más al área mayor que aquellas con un porcentaje provincial
importante. Longford (1999) extendió esta idea a vectores de variables
aleatorias de áreas menores, con la posibilidad de aprovechar la
correlación entre las mismas. En nuestro caso si la tasa de mortalidad
del grupo de edad entre x y x+4 del área i es mayor que la del área j,
que la correlación sea alta implicaría que en una edad contigua ocurra
lo mismo con más chance. Los cálculos seguidos en este trabajo en base a
esta idea, siguen el desarrollo metodológico de Assuncao (2005, pág. 543
y 544), que estima los parámetros por el método de los momentos para el
caso de tasas de fecundidad en Brasil.

\hypertarget{resultados}{%
\section{Resultados}\label{resultados}}

Para poner resultados usar

Estimaciones de esperanza de vida e intervalos de confianza en areas
menores en cada pronvincia.. Elegimos 4 .

\hypertarget{tables}{%
\subsection{Tables}\label{tables}}

\hypertarget{figures}{%
\subsection{Figures}\label{figures}}

\hypertarget{results-from-analyses}{%
\subsection{Results from analyses}\label{results-from-analyses}}

\hypertarget{conclusions}{%
\section{Conclusions}\label{conclusions}}

\hypertarget{posibles-impactos}{%
\subsection{Posibles impactos}\label{posibles-impactos}}

El plan de trabajo aquí propuesto plantea, a su vez, aportar a la
construcción de una base de datos para la evaluación y formulación de
políticas públicas, con el objeto de brindar a las autoridades
responsables en cada nivel geográfico (provinciales y municipales) de
herramientas cuyo diagnóstico permita identificar grupos de población
específicos para la ejecución de políticas públicas particulares, en
base a datos censales y estadísticas vitales.

Se propone utilizar la información construida como insumo información
para la Latin American Human Mortality Database
(\url{https://lamortalidad.org/}) y profundizar el estudio de los
efectos edad-periodo-cohorte de la mortalidad en Argentina, parcialmente
estudiados a nivel de total del país por Sacco
(\protect\hyperlink{ref-Sacco2016}{2016}\protect\hyperlink{ref-Sacco2016}{a})
y en la Ciudad de Buenos Aires por Grushka y Sacco
(\protect\hyperlink{ref-GrushkaSacco2017}{2017}), a fin de agregar una
observación adicional a las miradas transversales.

La investigación propuesta tiene, en el corto plazo, importancia
provincial, regional y local. La información derivada puede contribuir
al desarrollo de investigaciones en profundidad en el futuro por parte
de distintos usuarios y se espera que sea apta para la evaluación y
formulación de políticas públicas y para brindar a las autoridades
responsables en cada nivel geográfico y administrativo de herramientas
cuyo diagnóstico permita identificar grupos de población específicos
para la ejecución de políticas públicas particulares.

Siendo múltiples los lazos entre reproducción de población y
reproducción social y entre crecimiento económico y crecimiento
demográfico (Otero \protect\hyperlink{ref-Otero2007}{2007}), los
resultados de este trabajo están vinculada estrechamente como insumo
para el diseño de políticas públicas y en la consideración de la
mortalidad como componente necesario del desarrollo económico.

\parskip=5pt \parindent=0pt \spacing{1.0}

\appendix

\hypertarget{sec:append}{%
\section{Appendix}\label{sec:append}}

\hypertarget{descartes}{%
\section{Descartes}\label{descartes}}

\hypertarget{references}{%
\section{References}\label{references}}

\hypertarget{refs}{}
\leavevmode\hypertarget{ref-Bankirer2010}{}%
Bankirer, Mónica. 2010. ``La dinámica poblacional en tiempos del ajuste:
mortalidad y fecundidad''. En \emph{El costo social del ajuste
(Argentina 1976-2002)}, editado por Susana Torrado. Buenos Aires:
Edhasa.

\leavevmode\hypertarget{ref-CELADE2012}{}%
CELADE. 2012. ``La calidad de las estadísticas vitales en la América
Latina''. CEPAL/CELADE.

\leavevmode\hypertarget{ref-ChettyEtAl2016}{}%
Chetty, Raj, Michael Stepner, Sarah Abraham, Shelby Lin, Benjamin
Scuderi, Nicholas Turner, Augustin Bergeron, y David Cutler. 2016. ``The
Association Between Income and Life Expectancy in the United States,
2001-2014''. \emph{Clinical Review \& Education} 315 (16): 1750--66.

\leavevmode\hypertarget{ref-CohenPrestonCrimmins2011}{}%
Cohen, B., S.H. Preston, y E.M. Crimmins. 2011. \emph{International
Differences in Mortality at Older Ages: Dimensions and Sources}.
National Academies Press.
\url{https://books.google.com/books?id=cF8SvFAlHSYC}.

\leavevmode\hypertarget{ref-FernandezEtAl2008}{}%
Fernández, M., C. Guevel, H. Krupitzki, É. Marconi, y C. Massa. 2008.
\emph{Omisión de registro de nacimientos y muertes infantiles. Magnitud,
desigualdades y causas}. 1a ed. Buenos Aires: Organización Panamericana
de la Salud, Ministerio de Salud.

\leavevmode\hypertarget{ref-FreireEtAl2015}{}%
Freire, Queiroz, F. H. M. d. A. 2015. ``Mortality Estimates and
Construction of Life Tables for Small Areas in Brazil, 2010''.

\leavevmode\hypertarget{ref-Grushka2010}{}%
Grushka, Carlos O. 2010. ``¿Cuánto vivimos? ¿Cuánto viviremos?'' En
\emph{Dinámica de una ciudad: Buenos Aires, 1810-2010}, editado por A.
Lattes. Buenos Aires: Direccion General de Estadistica y Censos,
Gobierno de la Ciudad de Buenos Aires.

\leavevmode\hypertarget{ref-Grushka2014}{}%
---------. 2014. ``Casi un siglo y medio de mortalidad en la
Argentina...'' \emph{Revista Latinoamericana De Población} 8 (15,
Julio/Diciembre): 93--118.
\url{http://revistarelap.org/ojs/index.php/relap/article/view/14/13}.

\leavevmode\hypertarget{ref-GrushkaSacco2017}{}%
Grushka, Carlos O., y N. Sacco. 2017. ``La mortalidad de las cohortes en
la Ciudad de Buenos Aires''. Journal Article. \emph{Población de Buenos
Aires} 14 (25): 7--27.

\leavevmode\hypertarget{ref-INDEC-UNICEF2003}{}%
INDEC-UNICEF. 2003. \emph{Situación de los niños y adolescentes en la
Argentina 1990-2001}. Buenos Aires, Argentina: INDEC-UNICEF.

\leavevmode\hypertarget{ref-JaspersOrellana1994}{}%
Jaspers, D., y H. Orellana. 1994. ``Evaluación del uso de las
estadísticas vitales para estudios de causas de muerte en América
Latina''. \emph{Notas de Población} 60 (CELADE): 47--77.

\leavevmode\hypertarget{ref-LimaQueiroz2014}{}%
Lima, Everton Emanuel Campos de, y Bernardo Lanza Queiroz. 2014.
``Evolution of the deaths registry system in Brazil: associations with
changes in the mortality profile, under-registration of death counts,
and ill-defined causes of death''. \emph{Cadernos de Saúde Pública} 30
(8): 1721--30.
\url{https://doi.org/https://dx.doi.org/10.1590/0102-311X00131113}.

\leavevmode\hypertarget{ref-LimaQueirozSawyer2014}{}%
Lima, Everton Emanuel Campos de, Bernardo Lanza Queiroz, y Diana Oya
Sawyer. 2014. ``Método de estimação de grau de cobertura em pequenas
áreas: uma aplicação nas microrregiões mineiras''. \emph{Cadernos Saúde
Coletiva} 22 (4): 409--18.
\url{https://doi.org/https://dx.doi.org/10.1590/1414-462X201400040015}.

\leavevmode\hypertarget{ref-Mazzeo2005}{}%
Mazzeo, Victoria. 2005. ``¿Qué debemos mejorar en el registro de las
estadísticas vitales?'' \emph{Población de Buenos Aires} 2 (2): 69--78.

\leavevmode\hypertarget{ref-MesleVallin2011}{}%
Meslé, F., y J. Vallin. 2011. ``Historical Trends in Mortality''. En
\emph{International Handbook of Adult Mortality}, editado por Richard G.
Rogers y Eileen M. Crimmins, 2:9--47. International Handbooks of
Population. Springer Netherlands.
\url{https://doi.org/10.1007/978-90-481-9996-9_25}.

\leavevmode\hypertarget{ref-Mueller1978}{}%
Müller, María S. 1978. \emph{La mortalidad en Argentina. Evolución
histórica y situación en 1970}. Buenos Aires: CENEP-CELADE.

\leavevmode\hypertarget{ref-MuellerAccinelli1980}{}%
Müller, M. M., M. S.; Accinelli. 1980. ``Un hecho inquietante: la
evolución reciente de la mortalidad en la Argentina''. \emph{Cuaderno
del CENEP} 17.
\url{http://201.231.155.7/wwwisis/bv/cuadernos\%20cenep/CUAD\%2017.pdf}.

\leavevmode\hypertarget{ref-OeppenVaupel2002}{}%
Oeppen, Jim, y James W. Vaupel. 2002. ``Broken Limits to Life
Expectancy''. \emph{Science} 296: 1029--31.
\url{http://search.ebscohost.com/login.aspx?direct=true\&db=edsjsr\&AN=edsjsr.3076677\&lang=es\&site=eds-live}.

\leavevmode\hypertarget{ref-OMS2012}{}%
OMS. 2012. \emph{Estadísticas Sanitarias Mundiales}. Editado por OMS.
Organización Mundial de la Salud.

\leavevmode\hypertarget{ref-Otero2007}{}%
Otero, Hernán. 2007. ``El crecimiento de la población y la transición
demográfica''. En \emph{Población y bienestar en la Argetina del primero
al segundo Centenario. Una historia social del siglo XX}, editado por
Susana Torrado, I:339--67. Buenos Aires: Edhasa.

\leavevmode\hypertarget{ref-Pantelides1989}{}%
Pantelides, Edith Alejandra. 1989. \emph{La fecundidad argentina desde
mediados del siglo XX}. Buenos Aires: CENEP.

\leavevmode\hypertarget{ref-Poblacion2013}{}%
Población. 2013. ``Entrevista a Élida Marconi, Mercedes Fernández y
Carlos Guevel''. \emph{Población} 6 (Número 11): 23--27.

\leavevmode\hypertarget{ref-QueirozFreireGonzagaEtAl2017}{}%
Queiroz, B. L., Flávio Henrique Miranda de Araujo Freire, Marcos Roberto
Gonzaga, y Everton Emanuel Campos de Lima. 2017. ``Completeness of
death-count coverage and adult mortality (45q15) for Brazilian states
from 1980 to 2010''. \emph{Revista Brasileira de Epidemiologia} 20
(Suppl. 1): 21--33.
\url{https://doi.org/https://dx.doi.org/10.1590/1980-54972017000500033}.

\leavevmode\hypertarget{ref-RauEtAl2008}{}%
Rau, Roland, Eugeny Soroko, Domantas Jasilionis, y James W. Vaupel.
2008. ``Continued Reductions in Mortality at Advanced Ages''.
\emph{Population and Development Review} 34 (4): 747--68.
\url{http://www.jstor.org/stable/25434738}.

\leavevmode\hypertarget{ref-RecchinideLattes1975}{}%
Recchini de Lattes, Z. L., y Alfredo E. Lattes. 1975. \emph{La población
de Argentina}. Book. C.I.C.R.E.D. series. Buenos Aires: Talleres
Gráficos Zlotopioro, INDEC.

\leavevmode\hypertarget{ref-Ribotta2016}{}%
Ribotta, B. 2016. ``Estimaciones sub-nacionales de la cobertura de las
estadísticas vitales. Experiencias recientes en América Latina''. ALAP.

\leavevmode\hypertarget{ref-Riley2005}{}%
Riley, James C. 2005. ``The Timing and Pace of Health Transitions Around
the World''. \emph{Population and Development Review} 31 (4): 741--64.
\url{http://search.ebscohost.com/login.aspx?direct=true\&db=edsjsr\&AN=edsjsr.3401524\&lang=es\&site=eds-live}.

\leavevmode\hypertarget{ref-Rofman2007}{}%
Rofman, Rafael. 2007. ``Perspectivas de la población en el siglo XXI:
los segundos doscientos años''. En \emph{Población y Bienestar en
Argentina del Primero al Segundo Centenario. Una historia social del
siglo XX}, editado por Susana Torrado, II:603--32. Buenos Aires: Edhasa.

\leavevmode\hypertarget{ref-Sacco2016}{}%
Sacco, Nicolás. 2016a. ``¿Cuánto vivieron los nacidos a fines del siglo
XIX y cuánto vivirán los nacidos a fines del siglo XX?'' \emph{Notas de
Población} 103 (julio-diciembre): 73--100.

\leavevmode\hypertarget{ref-Sacco2016a}{}%
---------. 2016b. ``La calidad de los datos de mortalidad del Censo 2010
de Argentina''. \emph{Población y Salud en Mesoamérica} 14 (1): 1--24.
\url{https://doi.org/10.15517/psm.v14i1.25306}.

\leavevmode\hypertarget{ref-SetelEtAl2007}{}%
Setel, Philip W., Sarah B. Macfarlane, Simon Szreter, Lene Mikkelsen,
Prabhat Jha, Susan Stout, y Carla AbouZahr. 2007. ``A scandal of
invisibility: making everyone count by counting everyone''. \emph{The
Lancet} 370 (9598): 1569--77.
\url{https://doi.org/10.1016/S0140-6736(07)61307-5}.

\leavevmode\hypertarget{ref-Somoza1968}{}%
Somoza, Jorge L. 1968. \emph{Argentina, la mortalidad según tablas de
vida de 1914, 1946-1948 y 1959-1961}. Santiago: CELADE.
\url{http://books.google.com.ar/books?id=C2E-AAAAYAAJ}.

\leavevmode\hypertarget{ref-Somoza1970}{}%
---------. 1970. \emph{República Argentina, algunos efectos sociales y
económicos derivados de la baja de la mortalidad entre 1900 y 1960}.
Santiago: CELADE.
\url{http://books.google.com.ar/books?id=UGM-AAAAYAAJ}.

\leavevmode\hypertarget{ref-Somoza1971}{}%
---------. 1971. \emph{La mortalidad en la Argentina. Evolución
histórica y situación entre 1869 y 1960}. Buenos Aires: Editorial del
Instituto.

\leavevmode\hypertarget{ref-Somoza1972}{}%
---------. 1972. \emph{Argentina, un caso de descenso de la mortalidad}.
Population Reference Bureau (PRB), Programas Internacionales de
Población, Oficina Regional para América Latina.
\url{http://books.google.com.ar/books?id=UWM-AAAAYAAJ}.

\leavevmode\hypertarget{ref-Somoza1973}{}%
---------. 1973. ``La Mortalidad en la Argentina entre 1869 y 1960''.
\emph{Desarrollo Economico} 12 (No. 48 (Jan. - Mar., 1973)): 807--26.
\href{http://www.jstor.org/stable/3466306\%20.}{http://www.jstor.org/stable/3466306 .}

\leavevmode\hypertarget{ref-TimaeusDorringtonHill2013}{}%
Timæus, I., R. Dorrington, y K. Hill. 2013. ``Introduction to adult
mortality analysis''. En \emph{Tools for Demographic Estimation},
editado por Moultrie T., R. Dorrington, Hill A., Hill K., I. Timæus, y
B. Zaba, 191--94. Paris: International Union for the Scientific Study of
Population.
\href{\%20http://demographicestimation.iussp.org/content/generalized-growth-balance-method}{http://demographicestimation.iussp.org/content/generalized-growth-balance-method}.

\leavevmode\hypertarget{ref-Torrado1993}{}%
Torrado, Susana. 1993. \emph{Procreacion en la Argentina. Hechos e
ideas}. Buenos Aires: Ediciones de la Flor.
\url{http://books.google.com.ar/books?id=3EXdAAAAIAAJ}.

\leavevmode\hypertarget{ref-WilmothHoriuchi1999}{}%
Wilmoth, J.R., y S. Horiuchi. 1999. ``Rectangularization revisited:
Variability of age at death within human populations*''.
\emph{Demography} 36 (4): 475--95.
\url{https://doi.org/10.2307/2648085}.

%%% Doc-suffix

%%\printbibliography[heading=bibintoc]

%\printindex

%\printglossaries


%\printabbreviations


\printunsrtglossaries
\abbreviationsname{Abreviaturas}

%%\printglossary


\end{document}
